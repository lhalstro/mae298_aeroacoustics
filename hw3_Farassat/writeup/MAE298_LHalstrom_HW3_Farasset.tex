
\typeout{}\typeout{If latex fails to find aiaa-tc, read the README file!}
%


\documentclass[]{aiaa-tc}% insert '[draft]' option to show overfull boxes


\usepackage{amsmath}          % for formula writing (i.e. 'split', etc)
\usepackage{rotate}           %rotate/mirror images
\usepackage{cancel}           %draw lines through math to show "goes to zero"
\usepackage{xfrac}            %allows slated and side fractions
\usepackage{subcaption}       %allows captioning individual subfigures
\usepackage{multicol}         %enable environment with multiple columns
\usepackage[mode=buildnew]{standalone}% requires -shell-escape
  % compile with `pdflatex -shell-escape main` or `xelatex  -shell-escape main`


\usepackage{tikz}             %for creating vector graphics diagrams
\usetikzlibrary{backgrounds}  %put backgrounds behind tikz figures
\usetikzlibrary{calc}         %perform calculations within $$
\usetikzlibrary{positioning}  %position tikz elements using "right of, etc"
\usetikzlibrary{angles}       %label angles between lines with arcs
\usetikzlibrary{quotes}       %Put angle label in quotes
\usetikzlibrary{patterns}     %Patterns to fill shapes with






  \title{MAE 298 Aeroacoustics -- Homework \#3 \\ Generalized Differentiation and Farassat's Formulation}

\author{
  Logan D. Halstrom \\
  {\normalsize\itshape Graduate Student} \\
  {\normalsize\itshape Department of Mechanical and Aerospace Engineering} \\
  {\normalsize\itshape University of California, Davis, CA 95616}
       }


 % Define commands to assure consistent treatment throughout document
 \newcommand{\eqnref}[1]{(\ref{#1})}
 \newcommand{\class}[1]{\texttt{#1}}
 \newcommand{\package}[1]{\texttt{#1}}
 \newcommand{\file}[1]{\texttt{#1}}
 \newcommand{\BibTeX}{\textsc{Bib}\TeX}

%%%%%%%%%%%%%%%%%%%%%%%%%%%%%%%%%%%%%%%%%%%%%%%%%%%%%%%%%%%%%%%%%%%%%%%%
\begin{document}

\maketitle

% %%%%%%%%%%%%%%%%%%%%%%%%%%%%%%%%%%%%%%%%%%%%%%%%%%%%%%%%%%%%%%%%%%%%%%%%
% \begin{abstract}

% Abstract about homework

% \end{abstract}

%%%%%%%%%%%%%%%%%%%%%%%%%%%%%%%%%%%%%%%%%%%%%%%%%%%%%%%%%%%%%%%%%%%%%%%%
\section*{Nomenclature}

\begin{multicols}{2}

\begin{tabbing}
  XXX \= \kill% this line sets tab stop
  $L$                 \> Subscript for loading parameters \\
  $n$                 \> Surface normal direction \\
  $s$                 \> Surface coordinate variable \\
  $t$                 \> time \\
  $\vec{x}$           \> Observer location vector \\
  $r$                 \> Distance between source and observer \\
  $\hat{r}$           \> Unit vector between source and observer \\
  $\theta$            \> Angle between $\hat{r}$ and $\vec{x}$ \\
  $M$                 \> Mach number \\
  $W(r)$              \> Radial distribution of mean axial velocity \\
  $c$                 \> Speed of Sound\\
  $\overline{\rho}$   \> Mean density \\
  $\gamma$            \> Specific heat ratio \\
  $p$                 \> Pressure  \\
  $\widetilde{p}$     \> Pressure (Discontinuous across data surface)  \\
  $\overline{p}$      \> Mean pressure \\
  $p'$                \> Perturbation pressure \\
  $\Delta P$          \> Pressure difference from CFD solution \\
  $k$                 \> Wavenumber \\
  $\overline{\partial}$ \> Generalized derivative \\
  $\delta$            \> Delta function \\
  $\omega$            \> Wave oscillating frequency \\
  $i$                 \> Imaginary number $\sqrt{-1}$ \\
  $\exp$              \> Exponential ($e$) \\
  $E$                 \> Exponential term: $kz + n\theta -\omega t$ \\
  $\lambda$           \> Constant term in Bessel equation \\
  $J$                 \> First-order Bessel function \\
  $Y$                 \> Second-order Bessel function \\
  $H^{(n)}$           \> nth-order Hankel function \\
  $x$                 \> Placeholder variable for $\lambda r$ \\
  $A,B$               \> Arbitrary Bessel function constants \\
  $C,D$               \> Arbitrary Hankel constants \\
  $\vec{V}$           \> General velocity vector \\
  $V_r$               \> Velocity component in radial direction \\
  $\nu$               \> Constant velocity parameter \\
  $\chi$              \> Constant position parameter \\
  $\zeta$             \> Position of vortex sheet dividing inner/outer solution \\
  $+/-$               \> Outer/Inner solution, respectively \\




\end{tabbing}

\end{multicols}

%%%%%%%%%%%%%%%%%%%%%%%%%%%%%%%%%%%%%%%%%%%%%%%%%%%%%%%%%%%%%%%%%%%%%%%%
\section*{Overview} %%%%%%%%%%%%%%%%%%%%%%%%%%%%%%%%%%%%
%%%%%%%%%%%%%%%%%%%%%%%%%%%%%%%%%%%%%%%%%%%%%%%%%%%%%%%%%%%%%%%%%%%%%%%%


Explore techniques of generalized differentiations and solve FW-H loading term with Farassat's Formulation 1A




$r=|\vec{x} - \vec{y}|$


%%%%%%%%%%%%%%%%%%%%%%%%%%%%%%%%%%%%%%%%%%%%%%%%%%%%%%%%%%%%%%%%%%%%%%%%
\section{Problem 1 -- Generalized Differentiation of Wave Equation} \label{SecGenDiff}
%%%%%%%%%%%%%%%%%%%%%%%%%%%%%%%%%%%%%%%%%%%%%%%%%%%%%%%%%%%%%%%%%%%%%%%%

The acoustic wave equation without considering the source is expressed as follows:

\begin{equation} \label{AcousticWaveHomo}
\dfrac{1}{c^2}\dfrac{\partial^2p}{\partial t^2} - \nabla^2p = 0
\end{equation}

We can define a new function $\widetilde{p}$ using the imbedding technique as follows:
\begin{equation} \label{Ptilde}
\begin{split}
\widetilde{p} =
    \left\{ \begin{array}{lll}
        p, & f > 0 \\
        0, & f < 0
    \end{array} \right.
\end{split}
\end{equation}

\noindent where $f = 0$ describes the arbitrary moving body. Show that the wave equation whose sound is
generated by an arbitrary moving body (f=0) can be expressed as follows:

\begin{equation} \label{AcousticWaveBody}
\dfrac{1}{c^2}\dfrac{ \overline{\partial}^2 \widetilde{p} }{ \partial t^2}
    - \overline{\nabla}^2 \widetilde{p}
= -\left[ \dfrac{M_n}{c} \dfrac{\partial p}{\partial t} + p_n \right] \delta(f)
    -\dfrac{1}{c} \dfrac{\partial}{\partial t} \left[ M_n p \delta(f) \right]
    -\nabla \cdot \left[ p \vec{n} \delta(f) \right]
\end{equation}

\noindent where $\vec{n}$ is the unit normal vector on the surface and $p_n = \nabla p \cdot \vec{n}$. Now we can use the Green’s function of the wave equation in the unbounded space, the so-called free-space Green’s function, to find the unknown function $p(\vec{x},t)$   everywhere in space. The result is the Kirchhoff formula for moving surfaces.


%%%%%%%%%%%%%%%%%%%%%%%%%%%%%%%%%%%%%%%%%%%%%%%%%%%%%%%%%%%%%%%%%%%%%%%%
\subsection{Term 1}




%%%%%%%%%%%%%%%%%%%%%%%%%%%%%%%%%%%%%%%%%%%%%%%%%%%%%%%%%%%%%%%%%%%%%%%%
\section{Problem 2 -- Farassat Formulation 1A for Loading Noise} \label{SecFarassat}
%%%%%%%%%%%%%%%%%%%%%%%%%%%%%%%%%%%%%%%%%%%%%%%%%%%%%%%%%%%%%%%%%%%%%%%%


Farassat’s formulation 1 for the loading noise is given as

\begin{equation} \label{FarassatForm1Loading}
4\pi p_{L}' = \dfrac{1}{c} \dfrac{\partial}{\partial t}
      \int_{f=0} \left[ \dfrac{L_r}{r   (1 - M_r)} \right]_{ret} ds
    + \int_{f=0} \left[ \dfrac{L_r}{r^2 (1 - M_r)} \right]_{ret} ds
\end{equation}


\noindent where $L_r = \Delta P \vec{n} \cdot \hat{r} = \Delta P \cos\theta$. This formulation 1 is difficult to compute since the observer time differentiation is outside the integrals. A much more efficient and practical formulation can be derived by carrying the observer time derivate inside the integrals (formulation 1A). Show that formulation 1A for the loading noise becomes

\begin{equation} \label{FarassatForm1ALoadingEx}
4\pi p_{L}' = \dfrac{1}{c}
      \int_{f=0} \left[ \dfrac{\dot{L}_r}{r (1 - M_r)^2} \right]_{ret} ds
    + \int_{f=0} \left[ \dfrac{L_r - L_M}{r^2 (1 - M_r)} \right]_{ret} ds
+ \dfrac{1}{c} \int_{f=0} \left[
    \dfrac{ L_r[r \dot{M}_r + c(M_r - M^2)] }{r^2 (1 - M_r)^3} \right]_{ret} ds
\end{equation}

\noindent where $L_M = \vec{L} \cdot \vec{M} $.




%%%%%%%%%%%%%%%%%%%%%%%%%%%%%%%%%%%%%%%%%%%%%%%%%%%%%%%%%%%%%%%%%%%%%%%%
\section*{Conclusion}
%%%%%%%%%%%%%%%%%%%%%%%%%%%%%%%%%%%%%%%%%%%%%%%%%%%%%%%%%%%%%%%%%%%%%%%%

conclude



\end{document}


